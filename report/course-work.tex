\documentclass[a4paper,12pt,titlepage,finall]{article}

\usepackage[T1,T2A]{fontenc}    % форматы шрифтов
\usepackage[utf8x]{inputenc}    % кодировка символов, используемая в данном файле
\usepackage[russian]{babel}     % пакет русификации
\usepackage{cmap}               % search

\usepackage{geometry}           % для настройки размера полей
\usepackage{indentfirst}        % для отступа в первом абзаце секции
\usepackage{listings}           % код
\usepackage{xcolor}             % определение цветов
\usepackage{amssymb}            % символ пустого множества
% \usepackage{amsthm}             % доказательства

\newtheorem{definition}{Определение}
\newtheorem{theorem}{Теорема}
% \newtheorem{lemma}{Лемма}

\definecolor{purple}{rgb}{0.41, 0.21, 0.61}
\definecolor{green}{rgb}{0.01, 0.75, 0.24}

\newcommand{\code}[1]{\textcolor{gray}{\texttt{#1}}}

\lstset {
    language=c++,
    tabsize=4,
    basicstyle=\footnotesize\ttfamily,
    keywordstyle=\color{purple},
    stringstyle=\color{green},
    commentstyle=\color{gray},
    showstringspaces=false
}

\geometry{a4paper,left=30mm,top=30mm,bottom=30mm,right=30mm}
\setcounter{secnumdepth}{0}

\begin{document}
\begin{titlepage}
    \begin{center}
    {\small \sc Московский государственный университет \\имени М.~В.~Ломоносова\\
    Факультет вычислительной математики и кибернетики\\Кафедра математической кибернетики\\}
    \vfill
    {\Large \sc Курсовая работа}\\
    ~\\
    {\large \bf <<Эвристическая минимизация обобщенных полиномов>>}\\
    ~\\
    \end{center}
    \begin{flushright}
    \vfill {Выполнил:\\
    студент 318 группы\\
    Королёв~Ф.~И.\\
    ~\\
    Научный руководитель:\\
    к.ф.-м.н. Бухман~А.~В.}
    \end{flushright}
    \begin{center}
    \vfill
    {\small Москва\\2021}
    \end{center}
\end{titlepage}

\tableofcontents
\newpage

\section{Основные определения}

Рассмотрим данные понятия из области дискретной математики~\cite{discrete}.

\begin{definition}
    Пусть $E_2 = \{0, 1\}$ -- основное множество (исходный алфавит значений переменных), тогда $E^n_2 = \{(\alpha_1, \ldots, \alpha_n) | \forall i \  \alpha_i \in E_2\}$. Тогда всюду определенной булевой функцией назовем отображение $f(x_1, \ldots, x_n): \  E^n_2 \rightarrow E_2$.
\end{definition}
\begin{definition}
    Монотонной конъюнкцией от переменных $x_1, \ldots, x_n$ называется любое выражение вида $x_{i_1} \cdot x_{i_2} \cdot \ldots \cdot x_{i_s}$, где $s \geq 1, \  1 \leq i_j \leq n \  \forall j = \overline{1, s}$, все переменные различны ($i_j \neq i_k$, если $j \neq k$); либо просто 1.
\end{definition}
\begin{definition}
    Полиномом Жегалкина над $x_1, \ldots, x_n$ называется выражение вида
    $$K_1 \oplus K_2 \oplus \ldots \oplus K_l,$$
    где $l \geq 1$ и все $K_j$ суть различные монотонные конъюнкции над $x_1, \ldots, x_n$; либо константа 0.
\end{definition}
\begin{theorem}{Теорема Жегалкина}

    Любую функцию алгебры логики $f(x_1, \ldots, x_n)$ можно единственным образом выразить полиноном Жегалкина над $x_1, \ldots, x_n$.
\end{theorem}
\begin{definition}
    Обобщенным полиномом над $x_1, \ldots, x_n$ называется выражение вида
    $$K_1 \oplus K_2 \oplus \ldots \oplus K_l,$$
    где $l \geq 1$ и все $K_j$ суть различные монотонные конъюнкции над $x_1, \ldots, x_n$ или их отрицаниями; либо константа 0.
\end{definition}

\section{Введение}

Аналога теоремы Жегалкина для обобщенных полиномов не существует: можно найти несколько обобщенных полиномов, выражающих одну и ту же булеву функцию, например, функция алгебры логики $(0000\:1101)$ представима как $\overline x_3 \oplus x_1 \overline x_2 \oplus \overline x_2 x_3 \oplus \overline x_1 \overline x_2 x_3$ и как $\overline x_3 \oplus x_1 \overline x_2 \overline x_3$.

Отсюда вытекает интерес в поиске наименьшего с точки зрения количества конъюнктов обобщенного полинома, поскольку обычно они значительно короче аналогичной СДНФ, соответственно, потребуется меньше вычислительных мощностей для подсчета значения функции на определенном векторе значений.

Например, для $n$-мерной функции верхняя оценка количества конъюнктов в обобщенном полиноме равна $29 \cdot 2^{n - 7}$, при $n > 6$~\cite{bound}, для ДНФ же $2^{n - 1}$.

Сама задача минимизации обобщенных полиномов на текущий момент не имеет эффективного решения для произвольного количества переменных, однако научным сообществом публикуются работы с использованием эвристических алгоритмов.

В качестве предложенного в данной работе алгоритма выступает дискретный аналог покоординатного спуска~\cite{metopt}: к изначальному полиному прибавляются нулевые полиномы (простейший пример нулевого полинома -- $1 \oplus x_1 \oplus \overline x_1 \equiv 0$), если после сокращения конъюнктов ($x \oplus x \equiv 0$, а также $x \oplus 0 \equiv x$) удалось получить полином меньшей длины, то далее работа ведется с полученным полиномом.
Попытки продолжаются, пока при каждом проходе по всему базису нулевых полиномов удается сократить текущий полином хотя бы единожды, иначе алгоритм поиска останавливается.

\section{Постановка задачи}

Написать программу, входными данными которой служит обобщенный многочлен, выходными -- сокращенный программой обобщенный многочлен.

Сокращающий алгоритм должен опираться на описанный выше дискретный аналог покоординатного спуска.

Проанализировать работу программы:
\begin{itemize}
    \item понять, всегда ли получается найти наименьший обобщенный полином
    \item понаблюдать за скоростью работы и потреблением памяти программы
\end{itemize}

\section{Основная часть}

На языке C++ реализована программа, которая пытается эвристически решить поставленную задачу.

Обобщенные полиномы хранятся в программе в рамках следующей иерархии:

Класс переменной отвечает за выражения вида \code{x1, !x5, 0, 1}
\begin{lstlisting}
struct TVariable {
    uint32_t number;
    EState state;
}
\end{lstlisting}

Поле \code{number} отвечает за измерение переменной, для переменных вида \code{x2, !x3} хранит в явном виде числа \code{2, 3} соответственно.

Поле \code{state} хранит состояние соответствующей переменной, перечисление \code{EState} может хранить следующие состояния переменной:
\begin{lstlisting}
enum class EState {
    Empty,
    Positive,
    Negative
};
\end{lstlisting}
где \code{Empty} соответствует вхождению переменной в конъюнкт в качестве тождественной единицы, \code{Positive} в качестве самой переменной, \code{Negative} -- ее отрицания.

Константы \code{0, 1} выражаются как \code{TVariable\{ 0, EState::Empty \}, TVariable\{ 0, EState::Positive \}} соответственно.

Следующим уровнем абстракции является элементарная конъюнкция, описанная классом
\begin{lstlisting}
class TElementaryConjuction {
private:
    const uint32_t dimension;
    std::vector<TVariable> conjuction;
    mutable bool excess;
public:
    void normalize();
    // ...
};
\end{lstlisting}
где поле \code{dimension} описывает размерность пространства конъюнкта (нужно для отлавливания ошибок, где, например, в трехмерном пространстве задается конъюнкция $!x_1x_3x_4$), вектор \code{conjuction} задает саму конъюнкцию (последовательность \code{TVariable}, к которым последовательно применена операция $\land$), поле \code{excess} -- не влияющее на внешнее состояние объекта поле, необходимое для приведения обобщенного полинома к нормальной форме (в случае сокращение конъюнктов их поля \code{excess} помечаются \code{true}, т.е. они подлежат удалению).

Один из важных методов представлен функцией \code{normalize}, которая приводит конъюнкцию к нормальной форме (переменные предстают в конъюнкции в отсортированном порядке, если среди них есть повторения, то они сокращаются до единственного представителя; если в последовательности \code{TVariable} присутствуют константы, то в случае \code{1}, они удаляются, в случае \code{0} весь конъюнкт зануляется)

Сами обобщенные полиномы описаны классом
\begin{lstlisting}
class TPolynomial {
private:
    const uint32_t dimension;
    std::vector<TElementaryConjuction> polynomial;

    void normalize();
public:
    // ...
};
\end{lstlisting}
где поле \code{dimension}, аналогично с полем \code{TElementaryConjuction::dimension} служит для проверки ошибочного ввода, а вектор \code{polynomial} хранит в себе последовательность конъюнктов, к которым последовательно применяется операция $\oplus$.

Аналогично с \code{TElementaryConjuction::normalize} в \code{TPolynomial} представлен метод \code{normalize}, который сортирует между собой конъюнкции, сокращает повторяющиеся пары.

Действия программы представлены следующей последовательностью:
\begin{enumerate}
    \item считывание размерности булева пространства
    \item считывание обобщенного полинома из этого пространства
    \item генерация базиса нулевых полиномов для данного пространства
    \item последовательное прибавление элементов базиса к данному полиному, если результат суммы короче исходного, далее работаем с новым полиномом
    \item если при очередном проходе по базису нулевых полиномов не удалось сократить длину исходного полинома, алгоритм считается завершенным
\end{enumerate}

% \begin{lemma}
Мощность базиса нулевых полиномов в булевом пространстве размерности $n$ равна $3^n - 2^n$.
% \end{lemma}
% \begin{proof}

Базис будем строить с использованием дистрибутивности конъюнкции относительно исключающего или.
Представим себе элементарную конъюнкцию из $n$ множителей. На позиции i-того литерала может стоять один из трех вариантов: $\{0, 1, x_i \}$, где $0 \equiv 1 \oplus x_i \oplus \overline x_i$. Если хоть один литерал конъюнкции равен 0, то вся конъюнкция равна 0. В таком случае из $3^n$ вариантов уберем из рассмотрения все перестановки из множества $\{1, x_i \}$ для $\forall i \in \overline{1, n}$, отсюда нулевых многочленов, полученных раскрытием скобок, насчитывается $3^n - 2^n$.
% \end{proof}

В процессе генерации базиса нулевых полиномов генерируются все перестановки длиной $n$ (размерность пространства) с обязательным наличием хотя бы одного нулевого литерала, программно это выражено следующими сущностями:
\begin{lstlisting}
enum class EMode {
    Zero,
    One,
    Positive
};

    // basis generating function
    // ...
    std::vector<EMode> permutation;
    permutation.reserve(dimension);
    // generating correct permutation
\end{lstlisting}
Для каждой сгенерированной перестановки ее представление в виде нулевого полинома создавалось путем перемножения элементарных сущностей ($\{0, 1, x_i \}$, где $0 \equiv 1 \oplus x_i \oplus \overline x_i$) с помощью функций
\begin{lstlisting}
TPolynomial operator*(const TPolynomial &lhs, const TElementaryConjuction &rhs);
TPolynomial operator*(const TPolynomial &lhs, const TPolynomial &rhs);
\end{lstlisting}
В соответствии с леммой, мощность базиса нулевых многочленов, сгенерированных функцией
\begin{lstlisting}
std::vector<TPolynomial> generateZeroBasis(uint32_t dimension);
\end{lstlisting}
получалась равной $3^n - 2^n$.\\

Программа тестировалась на собственном наборе тестов.
Каждый из них будет представлен в виде:
\begin{itemize}
    \item вектор значений функции алгебры логики
    \item входной обобщенный полином
    \item последовательность найденных программой сокращений
    \item результат работы программы
    \item предложенное сокращение
    \item время работы программы
\end{itemize}

\textbf{Тест 1}
\\boolean function: $(1111\:0101)$
\\input: $x_1 \oplus x_2 \oplus x_3 \oplus \overline x_1 x_2 \oplus \overline x_1 \overline x_3 \oplus \overline x_2 \overline x_3 \oplus x_1 \overline x_2 x_3 \oplus \overline x_1 \overline x_2 \overline x_3$
\\matches:
\\$\overline x_2 \overline x_3 \oplus x_1 \overline x_2 \overline x_3 \oplus \overline x_1 \overline x_2 \overline x_3$
\\$x_2 \oplus x_1 x_2 \oplus \overline x_1 x_2$
\\$x_1 \oplus x_1 x_2 \oplus x_1 \overline x_2$
\\$x_1 \overline x_2 \oplus x_1 \overline x_2 x_3 \oplus x_1 \overline x_2 \overline x_3$
\\result: $x_3 \oplus \overline x_1 \overline x_3$
\\suggestion: $\overline x_1 \oplus x_1 x_3$
\\time: generating basis 0 ms, discrete coordinate descent 0 ms\\

\textbf{Тест 2}
\\boolean function: $(1010\:1100\;1011\:1101)$
\\input: $\overline x_2 \overline x_4 \oplus x_1 \overline x_2 x_3 \oplus x_1 x_3 \overline x_4 \oplus \overline x_1 \overline x_3 \overline x_4 \oplus x_2 \overline x_3 x_4 \oplus \overline x_2 \overline x_3 \overline x_4 \oplus x_1 x_2 x_3 x_4 \oplus x_1 x_2 \overline x_3 \overline x_4$
\\matches:
\\$\overline x_2 \overline x_4 \oplus \overline x_2 x_3 \overline x_4 \oplus \overline x_2 \overline x_3 \overline x_4$
\\result: $x_1 \overline x_2 x_3 \oplus x_1 x_3 \overline x_4 \oplus \overline x_1 \overline x_3 \overline x_4 \oplus x_2 \overline x_3 x_4 \oplus \overline x_2 x_3 \overline x_4 \oplus x_1 x_2 x_3 x_4 \oplus x_1 x_2 \overline x_3 \overline x_4$
\\suggestion: $x_1 x_3 \oplus x_2 \overline x_3 \oplus \overline x_1 \overline x_2 \overline x_4$
\\time: generating basis 9 ms, discrete coordinate descent 6 ms\\

\textbf{Тест 3}
\\boolean function: $(1011\:1101\;1111\:0000)$
\\input: $\overline x_1 \overline x_3 \oplus x_2 x_3 \oplus \overline x_2 \overline x_4 \oplus x_3 x_4 \oplus x_1 x_2 x_4 \oplus x_2 \overline x_3 \overline x_4 \oplus x_1 x_2 \overline x_3 \overline x_4 \oplus x_1 \overline x_2 \overline x_3 \overline x_4$
\\matches:
\\$x_2 \overline x_3 \overline x_4 \oplus x_1 x_2 \overline x_3 \overline x_4 \oplus \overline x_1 x_2 \overline x_3 \overline x_4$
\\result: $\overline x_1 \overline x_3 \oplus x_2 x_3 \oplus \overline x_2 \overline x_4 \oplus x_3 x_4 \oplus x_1 x_2 x_4 \oplus x_1 \overline x_2 \overline x_3 \overline x_4 \oplus \overline x_1 x_2 \overline x_3 \overline x_4$
\\suggestion: $x_3 \oplus x_1 x_2 x_4 \oplus \overline x_1 \overline x_3 x_4 \oplus x_2 x_3 x_4$
\\time: generating basis 2 ms, discrete coordinate descent 2 ms\\

\textbf{Тест 4}
\\boolean function: $(1010\:1101\;1010\:0100)$
\\input: $x_3 \oplus x_4 \oplus x_1 \overline x_2 \oplus \overline x_1 \overline x_4 \oplus x_1 x_2 \overline x_4 \oplus x_1 x_3 \overline x_4 \oplus x_1 \overline x_3 \overline x_4 \oplus x_1 x_2 x_3 x_4 \oplus \overline x_1 \overline x_2 \overline x_3 \overline x_4$
\\matches:
\\$x_1 \overline x_4 \oplus x_1 x_3 \overline x_4 \oplus x_1 \overline x_3 \overline x_4$
\\$\overline x_4 \oplus x_1 \overline x_4 \oplus \overline x_1 \overline x_4$
\\$1 \oplus x_4 \oplus \overline x_4$
\\$1 \oplus x_3 \oplus \overline x_3$
\\result: $\overline x_3 \oplus x_1 \overline x_2 \oplus x_1 x_2 \overline x_4 \oplus x_1 x_2 x_3 x_4 \oplus \overline x_1 \overline x_2 \overline x_3 \overline x_4$
\\suggestion: $x_1 \oplus \overline x_3 \oplus x_1 x_2 \overline x_3 x_4 \oplus \overline x_1 \overline x_2 \overline x_3 \overline x_4$
\\time: generating basis 2 ms, discrete coordinate descent 2 ms\\

\textbf{Тест 5}
\\boolean function: $(0010\:1000\;1010\:0000\;1000\:1000\;0000\:0000)$
\\input: $x_1 x_2 x_3 \overline x_4 \oplus x_1 x_2 x_3 \overline x_5 \oplus x_1 x_2 \overline x_3 x_4 \oplus x_1 \overline x_2 x_3 x_5$
\\matches: $\varnothing$
\\result: $x_1 x_2 x_3 \overline x_4 \oplus x_1 x_2 x_3 \overline x_5 \oplus x_1 x_2 \overline x_3 x_4 \oplus x_1 \overline x_2 x_3 x_5$
\\suggestion: $x_1 x_2 x_4 \oplus x_1 x_3 x_5$
\\time: generating basis 14 ms, discrete coordinate descent 3 ms\\

\textbf{Тест 6}
\\boolean function: $(1001\:0110\;0000\:0001\;1000\:0010\;0000\:0001)$
\\input: $x_4 x_5 \oplus x_1 x_2 x_3 x_4 \oplus x_1 x_2 x_4 x_5 \oplus x_1 \overline x_2 \overline x_3 x_4 \oplus x_1 x_3 x_4 x_5 \oplus \overline x_1 \overline x_2 \overline x_3 \overline x_4 \oplus x_2 x_3 x_4 x_5 \oplus \overline x_2 \overline x_3 x_4 x_5$
\\matches: $\varnothing$
\\result: $x_4 x_5 \oplus x_1 x_2 x_3 x_4 \oplus x_1 x_2 x_4 x_5 \oplus x_1 \overline x_2 \overline x_3 x_4 \oplus x_1 x_3 x_4 x_5 \oplus \overline x_1 \overline x_2 \overline x_3 \overline x_4 \oplus x_2 x_3 x_4 x_5 \oplus \overline x_2 \overline x_3 x_4 x_5$
\\suggestion: $x_1 x_2 x_4 \oplus x_1 \overline x_3 x_4 \oplus \overline x_1 x_2 x_4 x_5 \oplus \overline x_1 \overline x_2 \overline x_3 \overline x_4 \oplus \overline x_1 x_3 x_4 x_5$
\\time: generating basis 23 ms, discrete coordinate descent 5 ms\\

\textbf{Тест 7}
\\boolean function: $$(1100\:0000\;0000\:0000\;0100\:1011\;0000\:0000\\0000\:0000\;0000\:0000\;0100\:1011\;0000\:0000)$$
input: $x_1 x_2 x_4 \overline x_5 \oplus \overline x_2 \overline x_3 x_4 \overline x_5 \oplus \overline x_1 x_2 x_3 x_4 x_5 \oplus \overline x_1 x_2 x_3 x_4 \overline x_6 \oplus x_2 x_3 x_4 \overline x_5 x_6 \oplus x_1 x_2 x_3 x_4 x_5 x_6 \oplus x_1 x_2 x_3 x_4 \overline x_5 \overline x_6$
\\matches: $\varnothing$
\\result: $x_1 x_2 x_4 \overline x_5 \oplus \overline x_2 \overline x_3 x_4 \overline x_5 \oplus \overline x_1 x_2 x_3 x_4 x_5 \oplus \overline x_1 x_2 x_3 x_4 \overline x_6 \oplus x_2 x_3 x_4 \overline x_5 x_6 \oplus x_1 x_2 x_3 x_4 x_5 x_6 \oplus x_1 x_2 x_3 x_4 \overline x_5 \overline x_6$
\\suggestion: $x_1 x_2 x_4 \overline x_5 \oplus x_2 x_3 x_4 x_6 \oplus \overline x_2 \overline x_3 x_4 \overline x_5 \oplus x_2 x_3 x_4 \overline x_5 \overline x_6$
\\time: generating basis 94 ms, discrete coordinate descent 30 ms\\

\textbf{Тест 8}
\\boolean function: $$(0000\:1000\;0000\:0100\;1111\:1000\;0011\:01000011\:0111\;0001\:1011\;1100\:1011\;0000\:0111)$$
input: $\overline x_1 \overline x_6 \oplus x_2 x_4 \oplus x_3 x_5 \oplus x_3 x_6 \oplus \overline x_5 \overline x_6 \oplus x_1 \overline x_3 \overline x_6 \oplus x_1 \overline x_5 \overline x_6 \oplus \overline x_1 x_2 \overline x_3 \oplus x_2 x_3 x_4 \oplus x_2 x_3 \overline x_4 \oplus x_2 x_3 x_5 \oplus x_2 x_3 \overline x_6 \oplus x_3 x_5 \overline x_6 \oplus x_1 x_3 x_5 \overline x_6 \oplus \overline x_1 \overline x_3 \overline x_5 \overline x_6 \oplus x_2 x_3 x_4 x_5 \oplus x_2 \overline x_3 \overline x_5 \overline x_6 \oplus x_1 \overline x_2 x_3 \overline x_4 x_5 \overline x_6$
\\matches:
\\$\overline x_5 \overline x_6 \oplus x_1 \overline x_5 \overline x_6 \oplus \overline x_1 \overline x_5 \overline x_6$
\\$\overline x_1 \overline x_6 \oplus \overline x_1 x_5 \overline x_6 \oplus \overline x_1 \overline x_5 \overline x_6$
\\$x_3 x_5 \oplus x_2 x_3 x_5 \oplus \overline x_2 x_3 x_5$
\\$x_2 x_4 \oplus x_2 x_3 x_4 \oplus x_2 \overline x_3 x_4$
\\$x_3 x_5 \overline x_6 \oplus x_1 x_3 x_5 \overline x_6 \oplus \overline x_1 x_3 x_5 \overline x_6$
\\$\overline x_1 x_5 \overline x_6 \oplus \overline x_1 x_3 x_5 \overline x_6 \oplus \overline x_1 \overline x_3 x_5 \overline x_6$
\\$\overline x_1 \overline x_3 \overline x_6 \oplus \overline x_1 \overline x_3 x_5 \overline x_6 \oplus \overline x_1 \overline x_3 \overline x_5 \overline x_6$
\\$\overline x_3 \overline x_6 \oplus x_1 \overline x_3 \overline x_6 \oplus \overline x_1 \overline x_3 \overline x_6$
\\result: $x_3 x_6 \oplus \overline x_3 \overline x_6 \oplus \overline x_1 x_2 \overline x_3 \oplus x_2 x_3 \overline x_4 \oplus x_2 x_3 \overline x_6 \oplus x_2 \overline x_3 x_4 \oplus \overline x_2 x_3 x_5 \oplus x_2 x_3 x_4 x_5 \oplus x_2 \overline x_3 \overline x_5 \overline x_6 \oplus x_1 \overline x_2 x_3 \overline x_4 x_5 \overline x_6$
\\suggestion: $\overline x_2 \overline x_6 \oplus \overline x_1 x_2 \overline x_3 \oplus x_2 \overline x_3 x_4 \oplus \overline x_2 x_3 \overline x_5 \oplus x_2 x_3 x_4 \overline x_5 \oplus x_2 \overline x_3 x_5 \overline x_6 \oplus x_1 \overline x_2 x_3 \overline x_4 x_5 \overline x_6$
\\time: generating basis 92 ms, discrete coordinate descent 177 ms

Как видно из тестов, для 5 из них нашлись какие-то сокращения, и всего на 1 из них программа ответила минимальным обобщенным полиномом.

Также, в случае увеличения размерности булева пространства, начинает значительно расти требования по памяти: для 11-мерного пространства базис генерируется за 104.7 секунды и требует для своего хранения порядка 4.8 GB оперативной памяти.

\section{Полученные результаты}

Была написана программа для эвристического сжатия обобщенного полинома путем прибавления нулевых полиномов. Программа может сократить полином, но найти минимальный не может (так как удавалось придумать меньший полином в большинстве случаев), работает достаточно быстро, требовательна к памяти.

\begin{raggedright}
\addcontentsline{toc}{section}{Литература}
\begin{thebibliography}{99}
    \bibitem{discrete} Алексеев~В.\,Б., Поспелов~А.\,Д. Дискретная математика.\\Москва: Наука, 2002.
    \bibitem{bound} Gaidukov~A. Algorithm to derive minimum esop for 6-variable function.\\5th International Workshop on Boolean Problems, September 2002
    \bibitem{metopt} Васильев~Ф.\,П. Методы оптимизации.\\Москва: Издательство <<Факториал Пресс>>, 2002.
\end{thebibliography}
\end{raggedright}

\end{document}
